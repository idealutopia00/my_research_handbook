\chapter{智慧矿山数字孪生技术与平行智能研究综述}

随着新一代信息技术与煤炭工业的深度融合,煤矿智能化已成为实现煤炭工业高质量发展的核心技术支撑。当前,我国煤矿建设正处于从“单机自动化”向“智能感知、智能决策”跨越的关键时期。本文基于相关文献,从智慧矿山建设的必要性与顶层设计、数字孪生关键技术的发展现状、以及向平行智能演进的未来趋势三个方面进行综述。

\section{智能化建设的必要性与顶层设计}

煤炭开采面临着地质条件复杂、作业环境恶劣、设备群耦合关系强等严峻挑战。传统的“人-机”交互监控模式已难以满足安全高效生产的需求,亟需向“传感—机器认知—机器决策”的智能化模式转变\cite{DingEnJieJiYuWuLianWangDeKuangShanJiJieSheBeiZhuangTaiZhiNengGanZhiYuZhenDuan2023}。针对这一行业痛点,王国法等\cite{WangGuoFaMeiKuangZhiNengHuaMeiTanGongYeGaoZhiLiangFaZhanDeHeXinJiShuZhiCheng2019}明确提出了煤矿智能化的科学内涵,指出智慧煤矿的建设目标是实现煤炭资源的机器人化替代和全流程的智能管控。在具体的实施路径上,王国法等\cite{WangGuoFaZhiHuiMeiKuangYuZhiNengHuaKaiCaiGuanJianHeXinJiShuFenXi2019}进一步分析指出,为了解决多源异构数据统一表达及复杂围岩环境下的设备协同控制问题,必须构建包括多系统综合管理(MOS)、井下机器人群协同智慧在内的八大智能系统平台。这些顶层设计为后续的技术研发确立了“少人化、无人化”的根本导向。

\section{数字孪生驱动的智能化技术发展现状}

在明确了顶层设计后,如何实现物理矿山与数字世界的实时交互成为研究热点,数字孪生(Digital Twin, DT)技术因此被引入矿山领域。张帆等\cite{ZhangFanZhiHuiKuangShanShuZiLuanShengJiShuYanJiuZongShu2020}系统梳理了智慧矿山数字孪生技术架构,提出了基于“数字孪生+5G”的矿山数字孪生模型(MDTM),旨在通过虚实映射解决矿山开采过程中的非结构化环境感知难题。

在综采工作面这一核心场景中,葛世荣等\cite{GeShiRongShuZiLuanShengZhiCaiGongZuoMianJiShuJiaGouYanJiu2023}创新性地提出了数字孪生智采工作面(DTSMW)架构,将复杂的采煤过程归纳为环境、控制和能量三条信息流,并基于此构建了涵盖物理工作面、虚拟工作面及数据信息交互的完整技术体系。为了进一步提升控制的实时性与准确性,尤秀松等\cite{YouXiuSongZhiCaiGongZuoMianSanJiShuZiLuanShengQuDongKongZhiJiaGou2024}聚焦于采煤机、液压支架和刮板输送机的协同作业,建立了“三机”数字孪生驱动控制架构,利用知识模型解决数据拥堵问题,实现了复杂工况下的设备自适应控制。此外,针对被动交互机制的局限性,郭一楠等\cite{GuoYiNanZhiShiQuDongDeZhiCaiShuZiLuanShengZhuDongGuanKongMoShi2023}引入了知识驱动的主动管控模式,通过挖掘“环-机-物”孪生体间的隐含知识链,增强了虚拟空间对物理空间的管控鲁棒性。

除综采工作面外,掘进系统的智能化也在同步推进。杨健健等\cite{YangJianJianPingXingJueJinJiYuACPLiLunDeJueZhiMaoZhiNengKongZhiLiLunYuGuanJianJiShu2021}针对掘进巷道环境恶劣、群体设备难以协同的问题,提出了基于ACP(人工社会、计算实验、平行执行)理论的“平行掘进”思想,利用自适应动态规划多智能体方法,为掘-支-锚系统的智能控制提供了新的理论框架。

值得注意的是,上述理论已开始在实际工程中落地验证。李浩荡等\cite{LiHaoDangJiYuShuZiLuanShengYuGuiHuaFangMeiXieTongDeMeiKuangZhiNengKaiCaiJiShu2026}于2026年发布的最新研究显示,基于数字孪生与规划放煤协同的智能开采系统已在新疆乌东煤矿成功应用。该系统通过RBF神经网络动态博弈算法与多源数据融合,使生产系统自动化率达到95\%,规划放煤准确率达到93\%,有力证明了数字孪生技术在复杂地质条件下的实战价值。

\section{从数字孪生到平行智能的未来展望}

尽管数字孪生技术在矿山感知与控制方面取得了显著进展,但现有的应用多侧重于对物理系统的精准映射与实时监控,缺乏对未来场景的预测与引导能力。针对这一问题,陈龙等\cite{ChenLongPingXingKuangShanCongShuZiLuanShengDaoKuangShanZhiNeng2021}提出了从“数字孪生”向“平行矿山”演进的学术观点。未来的智慧矿山将不再局限于对物理世界的被动反映,而是通过构建人工矿山系统进行计算实验,产生海量虚拟数据以训练智能模型,最终通过平行执行实现对物理矿山的引导与优化。

与此同时,丁恩杰等\cite{DingEnJieKuangShanXinXiHuaFaZhanJiYiShuZiLuanShengWeiHeXinDeZhiHuiKuangShanGuanJianJiShu2023}指出,未来的核心竞争力将在于知识服务体系的构建。通过将矿山运行机理、经验知识与大数据分析深度融合,构建可信的数字孪生模型,并结合边缘计算与云边协同机制,将进一步提升智能装备的即时响应能力。综上所述,未来的研究将更加注重机理模型与数据驱动的深度融合,向着“虚实融合、以虚控实、平行智能”的终极目标迈进。·