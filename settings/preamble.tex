\ProvidesFile{preamble.tex}

% --- 基础编码与字体 ---
% 如果使用 Overleaf,推荐编译器选择 XeLaTeX
\usepackage[UTF8]{ctex} % 支持中文,关键包
\usepackage[utf8]{inputenc}
\usepackage[T1]{fontenc}

% --- 页面布局 ---
\usepackage{geometry}
\geometry{a4paper, left=2.5cm, right=2.5cm, top=3cm, bottom=3cm}

% --- 数学公式 ---
\usepackage{amsmath, amssymb, amsthm, bm}

% --- 图形与颜色 ---
\usepackage{graphicx}
\usepackage{xcolor}
\usepackage{float}
\usepackage{booktabs} % 三线表

% --- 参考文献管理 (BibLaTeX) ---
\usepackage[backend=biber, style=ieee, sorting=none]{biblatex}
\addbibresource{bib/uncategorized.bib} % 导入同级目录下的 bib 文件
\addbibresource{bib/0.bib} % 导入同级目录下的 bib 文件

% --- 超链接设置 ---
\usepackage{hyperref}
\usepackage{bookmark}
\hypersetup{
    colorlinks=true,
    linkcolor=blue,
    filecolor=magenta,      
    urlcolor=cyan,
    citecolor=red, % 引用颜色设为红色,方便查看
    pdftitle={Research Survey},
}

% --- 代码块设置 (适合写算法) ---
\usepackage{listings}
\lstset{
    basicstyle=\ttfamily\small,
    breaklines=true,
    frame=single,
    numbers=left,
    numberstyle=\tiny\color{gray},
    keywordstyle=\color{blue},
    commentstyle=\color{green!60!black}
}

% --- 开源项目归档专用设置 ---

% 1. 引入必要的包
\usepackage{tcolorbox}
\usepackage{fontawesome5} % 需要安装字体包,或者在 Overleaf 上直接可用
\usepackage{xcolor}

% 2. 定义项目卡片样式
\newtcolorbox{repocard}[2][]{
    colback=gray!5!white,      % 背景色:极淡的灰色
    colframe=black!70!white,   % 边框色:深灰
    title={\faGithub\ \textbf{#2}}, % 标题带 GitHub 图标
    coltitle=white,
    fonttitle=\bfseries\large,
    #1 % 允许传入额外参数
}

% 3. 定义一个快速命令 \opensource
% 参数: {项目名}{URL}{主要语言/框架}{简介}
\newcommand{\opensource}[4]{
    \begin{repocard}{#1}
        \textbf{URL:} \url{#2} \\
        \textbf{Stack:} \texttt{#3} \\
        \textbf{Description:} #4
    \end{repocard}
}